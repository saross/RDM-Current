% Latex template: https://github.com/mqTeXUsers/Macquarie-University-Beamer-Theme

% Slide Masters:

% Title
% Text
% 2 column
% Full-image
% Bibliography
% Closing
 
\documentclass[aspectratio=169, 12pt]{beamer} % Aspect ratio
% https://tex.stackexchange.com/a/14339/5483 
% Possible values: 1610, 169, 149, 54, 43 and 32.
% 169 = 16:9

\PassOptionsToPackage{table}{xcolor}    %https://tex.stackexchange.com/a/5365/5483

\usetheme{macquarie}
\usepackage{multicol} % https://tex.stackexchange.com/a/396018/5483

\usepackage{svg} %https://github.com/mrpiggi/svg 

\usepackage[english]{babel}       % Set language
% \usepackage[utf8x]{inputenc}      % Set encoding
\usepackage{colortbl}
\mode<presentation>           % Set options
{
  \usetheme{default}          % Set theme
  \usecolortheme{default}         % Set colors
  \usefonttheme{default}          % Set font theme
  \setbeamertemplate{caption}[numbered] % Set caption to be numbered
}

% Uncomment this to have the outline at the beginning of each section highlighted.
%\AtBeginSection[]
%{
%  \begin{frame}{Outline}
%    \tableofcontents[currentsection]
%  \end{frame}
%}

\usepackage{graphicx}         % For including figures
\usepackage{booktabs}         % For table rules
\usepackage{hyperref}         % For cross-referencing


\usepackage{enumitem} % https://tex.stackexchange.com/a/2292/5483

%https://tex.stackexchange.com/a/371844/5483
\setbeamerfont{bibliography entry author}{size=\tiny}
\setbeamerfont{bibliography entry title}{size=\tiny}
\setbeamerfont{bibliography entry location}{size=\tiny}
\setbeamerfont{bibliography entry note}{size=\tiny}
\setbeamerfont{bibliography item}{size=\tiny}

%https://tex.stackexchange.com/q/333587/5483
%TODO SHAWN REPLACE OSF URL
%\setbeamertemplate{footline}{\strut~\texttt{https://osf.io/v5jp7/}\hfill\insertframenumber~/~\inserttotalframenumber\strut~~~}

\title{RDM in context} % Presentation title
\author{Shawn A Ross}               % Presentation author
\institute{Office of the Deputy Vice-Chancellor (Research)}         % Author affiliation
\date{\today}                 % Today's date  
\begin{document}

% Title page
% This page includes the informations defined earlier including title, author/s, affiliation/s and the date
% \begin{frame}[noframenumbering]

\maketitle

  
% \end{frame}

% Outline
% This page includes the outline (Table of content) of the presentation. All sections and subsections will appear in the outline by default.
\begin{frame}{The legal, ethical, funder, and publisher landscape}
  \tableofcontents
\end{frame}

% The following is the most frequently used slide types in beamer
% The slide structure is as follows:
%
%\begin{frame}{<slide-title>}
% <content>
%\end{frame}

\section{Transparency and reproducibility}

\begin{frame}{The `reproducibility crisis'}
  For nearly a decade the reproducibility crisis has featured in the scientific literature \cite{Jasny2011-bw, Baker2016-cf, Munafo2017-bj}. Low reproducibility rates have emerged from large-scale studies:
    \begin{itemize}[label=\textbullet]
        \item Results from only 39\% of psychology studies could be reproduced \cite{Open_Science_Collaboration2015-vf}.
        \item Even lower reproducibility rate in biomedical research \cite{Begley2012-xt,Prinz2011-za}.
    \end{itemize}
\end{frame}

\begin{frame}{Perceptions of the reproducibility crisis}
  \begin{figure}[H]
    \centering
        \includegraphics[height=.7\textheight]{figures/reproducibility-graphic-online1.jpeg}
        \caption{Is there a reproducibility crisis? \cite{Baker2016-cf}}
        \label{fig:figure3}
  \end{figure}
\end{frame}

\begin{frame}{Motivation: Preserving data}
 \begin{figure}[H]
    \centering
        \includegraphics[height=.75\textheight]{figures/Missing-Data.png}
        \caption{\cite{Vines2014-zr}}
        \label{fig:vines2014}
 \end{figure}
\end{frame}

\begin{frame}{The response: improved rigour and transparency}
  Key guidelines to good practice:
    \begin{itemize}[label=\textbullet]
        \item Findable, Accessible, Interoperable, and Reusable (FAIR) data \cite{Wilkinson2016-mr, Go-fair2017-vs}.
        \item Transparency and Openness Promotion (TOP) guidelines \cite{Nosek2015-wm}.
        \item Data transparency toolkit \cite{Perkel2018-rw}.
    \end{itemize}
\end{frame}

\begin{frame}{The response: from guidelines to mandates}
  Mandates for transparency or reproducibility:
    \begin{itemize}[label=\textbullet]
        \item Nature: Transparency Upgrade \cite{Nature2017-lq}.
        \item Nature: FAIR data in Earth science \cite{Nature2019-ng}.
        \item Copernicus: FAIR data in atmospheric sciences \cite{Van_Edig2018-bu}.
        \item TOP Guidelines signatories include publishers representing 1000+ journals, as well as professional organisations and major private foundations that fund research \cite{Cos2019-mr}.
        \item Funders' data policies \cite{Dcc2019-jn}
        \item Not just the natural sciences: AJPS requires data and code \cite{Jacoby2017-lw, Ajps2015-ex}; ALLEA open consultation on FAIR data in the humanities \cite{Allea2019-vq}.
        
    \end{itemize}
\end{frame}


% https://tex.stackexchange.com/a/2292/5483
% https://ctan.org/pkg/enumitem?lang=en

\begin{frame}{Level 2 TOP Guidelines for authors (excerpt)}
  
    \begin{enumerate}[label=\arabic*.]
        \setcounter{enumi}{1}
        % This increments the enumerate counter by 1.
        
        \item Authors using original data must:
        \begin{enumerate}[label=\alph*.]

            \item make the data available at a trusted digital repository [...]
            \item include all variables, treatment conditions, and observations described in the manuscript.
            \item provide a full account of the procedures used to collect, preprocess, clean, or generate the data.
            \item provide program code, scripts, codebooks, and other documentation sufficient to precisely reproduce all published results.
            \item provide research materials and description of procedures necessary to conduct an independent replication of the research.
        \end{enumerate}
    \end{enumerate}
    \cite{Osf2014-pf}
\end{frame}

\begin{frame}{TOP Guidelines: publisher adoption}
  \begin{figure}[H]
    \centering
        \includegraphics[height=.7\textheight]{figures/TOP-landscape.png}
        \caption{The Landscape of Open Data Policies \cite{Mellor2018-bf}}
        \label{fig:figure2}
  \end{figure}
\end{frame}

\begin{frame}{TOP Guidelines: funder endorsement}
  Private funders have endorsed via the Open Funders Research Group:
    \begin{itemize}[label=\textbullet]
        \item Alfred P. Sloan Foundation
        \item American Heart Association
        \item Bill and Melinda Gates Foundation
        \item Howard Hughes medical Institute
        \item John Templeton Foundation
        \item Laura and John Arnold Foundation
        \item Open Society Foundations
        \item Robert Wood Johnson Foundation
        \item Wellcome Trust
        \item and six more \cite{Ofrg2019-pq}
    \end{itemize}
\end{frame}

\begin{frame}{Other Funder data policies}
  \begin{figure}[H]
    \centering
        \includegraphics[height=.7\textheight]{figures/DCC-Funders.png}
        \caption{Overview of funders' data policies \cite{Dcc2019-jn}}
        \label{fig:Dcc2018}
  \end{figure}
\end{frame}

\section{Data sharing and data ethics}

\begin{frame}{Data sharing in the NHMRC Statement}
    The NHMRC `strongly encourages' data sharing in the National Statement on Ethical Conduct in Human Research and their Open Access Policy. \cite{Nhmrc2018-sj, Nhmrc2018-vn} \par
    National Statement 3.1.50 \par
    In the absence of justifiable ethical reasons (such as respect for cultural ownership or unmanageable risks to the privacy of research participants) and to promote access to the benefits of research, researchers should collect and store data or information generated by research projects in such a way that they can be used in future research projects. Where a researcher believes there are valid reasons for not making data or information accessible, this must be justified.
\end{frame}

\begin{frame}{NHMRC Open Access Policy 2018 changes}
    Key changes to the Open Access Policy (15 January 2018) \par
    Research data and metadata (2.2) \par
    NHMRC now strongly encourages researchers to take reasonable steps to share research data and associated metadata arising from NHMRC supported research.\par
    FAIR principles (2.7) \par
    Reference to the Australian FAIR principles (Findable, Accessible, Interoperable, Reusable) when publishing research literature and sharing data has been made.
\end{frame}

\begin{frame}{NHMRC Open Access Policy data sharing}
    Medatdata (4.1) \par
    The metadata for the peer-reviewed publication must be made openly accessible via an institutional repository as soon as possible but no later than 3 months from the date of publication. \par
    Data (4.2) \par
    NHMRC acknowledges the importance of making research data publicly accessible and therefore strongly encourages researchers to consider the reuse value of their data and to take reasonable steps to share research data and associated metadata arising from NHMRC supported research.
\end{frame}

\section{Legal compliance}

\begin{frame}{NSW, Australian, international regulations}
    \begin{itemize}[label=\textbullet]
        \item NSW General Retention and Disposal Authority GDA 23; data associated with `significant' research or researchers must be kept forever (23.6.1) \cite{Nsw2015-kv}
        \item NSW Privacy and Personal Information Protection Act 1998 No 133, esp. Part 2, Division 1, Section 19, which flags indicators of high sensitivity and establishes data sovereignty.\cite{Nsw1998-mw} Compare the (Australian) Privacy Act 1988, esp. Part II, Division 1, Section 6 `Sensitive Information' and Schedule 1, and `Australian Privacy Principles', Section 8, which covers some university-controlled entities. \cite{Ag2017-oz,Oaic2019-ng}
        \item NSW Notifiable Data Breach guidance \cite{Ipc_nsw2018-yr}; see also the Australian Notifiable Data Breaches scheme \cite{Oaic2019-dq}
        \item EU General Data Protection Regulation \cite{Gdpr2019-ee}
    \end{itemize}
\end{frame}

\section{Compliance and beyond}

\begin{frame}{What does this mean? Are we ready?}
  Emerging good practice - and publisher and funder policies - mean:
    \begin{itemize}[label=\textbullet]
        \item Comprehensive, FAIR datasets will be deposited in domain-specific repositories. Data, and especially metadata, quality will be higher.
        \item Data will be captured digitally as early in research as possible, and provenance / version history maintained.
        \item Research approach, processes, and procedures will be documented.
        \item Data processing and analysis will use code (not Excel or ARCGIS!) 
        \item Code will be documented and published for reuse.
        \item Further steps taken for analytical reproducibility (use of OSS, version control, automation, containerisation, etc.). 
    \end{itemize}
\end{frame}

\begin{frame}{Beyond compliance: large-scale research}
    The same approaches that facilitate transparency and reproducibility support the kind of scalable and synthetic research that can address archaeological `grand challenges'. \cite{Kintigh2014-ub}
        \begin{itemize}[label=\textbullet]
            \item Paper data capture and manual digitisation and cleaning don't scale.
            \item Email and desktop software don't scale.
    \end{itemize}
\end{frame}

\begin{frame}{Scalable approaches to data and analysis}
  \begin{figure}[H]
    \centering
        \includegraphics[height=.7\textheight]{figures/Ocean-Health-Index.jpg}
        \caption{Better science in less time, illustrated by the Ocean Health Index project. \cite{Stewart_Lowndes2017-lj}}
        \label{fig:figure14}
  \end{figure}
\end{frame}

\section{References}

% \begin{frame}[allowframebreaks]{References}
  
%   \bibliography{references}
%   \bibliographystyle{apalike}
% \end{frame}

\begin{multicols}{2}[]
\bibliography{references}
\bibliographystyle{apalike}
\end{multicols}

% \bibliographystyle{apalike}

 
% Adding the option 'allowframebreaks' allows the contents of the slide to be expanded in more than one slide.
% The "1" comes from the outer theme"

\begin{frame}{Thank you!}

%This presentation is available at:
%\texttt{https://osf.io/v5jp7/}

Source code for this presentation is available at: 
\texttt{https://github.com/saross/RDM-Current}.

This work is licensed under a Creative Commons Attribution 4.0 International License.

\end{frame}



\end{document}
